\section{Research Experience}

\cventry{September 2016 - Present}{Professor David Brooks}{Using Near Data Processing Techniques in SSDs to Accelerate Hybrid Database Systems} {Harvard University}{}
  { Investigating using on-board Flash Translation Layer (FTL) compute resources to remap memory regions and pack more useful data into memory pages transferred over PCIe. Classical database systems must typically choose a single physical data layout for persistent storage, deciding between read performance and write performance. With on-the-fly translation within a Solid-State Drive (SSD), the higher internal Flash bandwidth can be leveraged to provide higher effective bandwidth over PCIe for regular but non-contiguous database access patterns. This will then provide more flexibility to the physical layout design decision and provide higher bandwidth to more balanced read/write workloads. \\
  }

\cventry{January 2015 - June 2015}{Dr. Vilas Sridharan}{Continued Development of the AMD Research AVF Analysis Infrastructure}{AMD Research}{}
  { Further developed the AMD Research internal AVF analysis infrastructure for the US Department of Energy Exascale supercomputing initiative research deliverables. Worked on extending Program Vulnerability Factor analysis to highly parallel applications. Collaborated on low-voltage reliability analysis techniques. \\
  }

\cventry{January 2013 - August 2013}{Dr. Vilas Sridharan}{Development of the AMD Research AVF Analysis Infrastructure}{AMD Research}{}
  { Developed an Architectural Vulnerability Factor (AVF) analysis infrastructure attached to an internal variant of the gem5 architectural simulator for Heterogeneous Systems Architecture Advanced Processing Units (HSA APUs). Made significant modifications to support an APU environment, implemented logical masking analysis for the HSA Intermediate Language (HSAIL), and enhanced the modeling to include new hardware structures (L1/L2 Cache, GPU Vector Register File). Developed novel modeling techniques to support spatial multi-bit faults. Contributed towards US Department of Energy Exascale supercomputing initiative research deliverables. \\
  }

\cventry{May 2012 - August 2012}{Professor David Kaeli}{Simulation of a Southern Islands GPU}{Northeastern University}{}
  { Implemented support for the Southern Islands GPU architecture on the heterogeneous simulation framework Multi2Sim. Implemented the disassembly and functional emulation of 121 instructions from the Southern Islands Instruction Set Architecture covering 16 benchmarks in the AMD APPSDK-2.5. Took part in the initial development of the cycle accurate architectural model and development of the Southern Islands user manual. \\
  }
